%
% LaTeX Helper Packages
%

% General Packages
\usepackage[plain]{algorithm}
\usepackage{algpseudocode}
\usepackage{calc}
\usepackage{xcolor}
\usepackage{xparse}

% Math Scripts/Symbols Packages
\usepackage{amsfonts}
\usepackage{amsmath}
\usepackage{amssymb}
\usepackage{amsthm}
\usepackage{bm}
\usepackage{calligra}
\usepackage{cancel}
\usepackage{cases}
\usepackage[per-mode=fraction]{siunitx}
\usepackage{fontspec}
\usepackage{polyglossia}
\setmonofont{IosevkaTerm Nerd Font Mono}
\usepackage[version=4]{mhchem}
\usepackage{padcount}

% Figure Packages
\usepackage{caption}
\usepackage{graphicx}
% \usepackage{subcaption}
\usepackage{tikz}
\usetikzlibrary{automata,positioning}

% Array and Table Packages
\usepackage{array}
\usepackage{longtable}
\usepackage{booktabs}
\usepackage{colortbl}

% Links/Code Packages
\usepackage{hyperref}
\usepackage{minted}
\usepackage[skins,minted]{tcolorbox}

% Nomenclature Package
\usepackage{nomencl}

%
% Engineering/Math Helper Commands
%

% Such That
\newcommand{\st}{\text{ s.t. }}
% With Regards To
\newcommand{\wrt}{\text{ w.r.t. }}

% Useful for algorithms
\newcommand{\alg}[1]{\textsc{\bfseries \footnotesize #1}}
\NewDocumentCommand{\fn}{s m o}{\, \IfBooleanTF{#1}{\mathrm{#2}}{\ensuremath{#2}}\IfValueT{#3}{\!\left(#3\right)}}

% For derivatives
\newcommand{\drv}[2]{\frac{\mathrm{d}#1}{\mathrm{d}#2}}
\newcommand{\ddrv}[2]{\frac{\mathrm{d}^{2}#1}{\mathrm{d}{#2}^{2}}}
\newcommand{\deriv}[2]{\frac{\mathrm{d}}{\mathrm{d}#1}\!\left(#2\right)}
\newcommand{\dderiv}[2]{\frac{\mathrm{d}^{2}}{\mathrm{d}{#1}^{2}}\!\left(#2\right)}

% For partial derivatives
\newcommand{\pdrv}[2]{\frac{\partial#1}{\partial#2}}
\newcommand{\pddrv}[2]{\frac{\partial^{2}#1}{\partial{#2}^{2}}}
\newcommand{\pderiv}[2]{\frac{\partial}{\partial#1}\!\left(#2\right)}
\newcommand{\pdderiv}[2]{\frac{\partial^{2}}{\partial{}{#1}^2}\!\left(#2\right)}

% Probability commands: Expectation, Variance, Covariance, Bias
\NewDocumentCommand{\E}{o}{\,\mathrm{E}\IfValueT{#1}{\!\left(#1\right)}}
\NewDocumentCommand{\Var}{o}{\,\mathrm{Var}\IfValueT{#1}{\!\left(#1\right)}}
\NewDocumentCommand{\Cov}{o}{\,\mathrm{Cov}\IfValueT{#1}{\!\left(#1\right)}}
\NewDocumentCommand{\Bias}{o}{\,\mathrm{Bias}\IfValueT{#1}{\!\left(#1\right)}}
\NewDocumentCommand{\Prob}{o}{\,\mathrm{Pr}\IfValueT{#1}{\!\left(#1\right)}}

% Col and Row Vectors
\newcommand{\bvect}[1]{\ensuremath{\begin{bmatrix}#1\end{bmatrix}}}
\newcommand{\pvect}[1]{\ensuremath{\begin{pmatrix}#1\end{pmatrix}}}

% For writing vectors
\let\oldhat\hat{}
\let\oldvec\vec{}
\renewcommand{\vec}[1]{\oldvec{\mathbf{#1}}}
\newcommand{\vecb}[1]{\mathbf{#1}}
\newcommand{\vech}[1]{\oldhat{\mathbf{#1}}}

\newcommand{\cpvect}[2]{\begin{pmatrix} #1 \\ #2 \end{pmatrix}}
\newcommand{\cptvect}[3]{\begin{pmatrix} #1 \\ #2 \\ #3 \end{pmatrix}}
\newcommand{\cbvect}[2]{\begin{bmatrix} #1 \\ #2 \end{bmatrix}}
\newcommand{\cbtvect}[3]{\begin{bmatrix} #1 \\ #2 \\ #3 \end{bmatrix}}
\newcommand{\rpvect}[2]{\begin{pmatrix} #1 & #2 \end{pmatrix}}
\newcommand{\rptvect}[3]{\begin{pmatrix} #1 & #2 & #3 \end{pmatrix}}
\newcommand{\rbvect}[2]{\begin{bmatrix} #1 & #2 \end{bmatrix}}
\newcommand{\rbtvect}[3]{\begin{bmatrix} #1 & #2 & #3 \end{bmatrix}}
\newcommand{\vect}[2]{\langle{} #1, #2 \rangle{}}
\newcommand{\tvect}[3]{\langle{} #1, #2, #3 \rangle{}}
\newcommand{\qvect}[4]{\langle{} #1, #2, #3, #4 \rangle{}}

% For sin and cos and tangent, etc
\let\oldsin\sin{}
\renewcommand{\sin}[1]{\oldsin\!\left(#1\right)}
\let\oldcos\cos{}
\renewcommand{\cos}[1]{\oldcos\!\left(#1\right)}
\let\oldtan\tan{}
\renewcommand{\tan}[1]{\oldtan\!\left(#1\right)}
\let\oldcsc\csc{}
\renewcommand{\csc}[1]{\oldcsc\!\left(#1\right)}
\let\oldsec\sec{}
\renewcommand{\sec}[1]{\oldsec\!\left(#1\right)}
\let\oldcot\cot{}
\renewcommand{\cot}[1]{\oldcot\!\left(#1\right)}
\let\oldsinh\sinh{}
\renewcommand{\sinh}[1]{\oldsinh\!\left(#1\right)}
\let\oldcosh\cosh{}
\renewcommand{\cosh}[1]{\oldcosh\!\left(#1\right)}
\let\oldtanh\tanh{}
\renewcommand{\tanh}[1]{\oldtanh\!\left(#1\right)}
\let\oldcoth\coth{}
\renewcommand{\coth}[1]{\oldcoth\!\left(#1\right)}
\let\oldarcsin\arcsin{}
\renewcommand{\arcsin}[1]{\oldarcsin\!\left(#1\right)}
\let\oldarccos\arccos{}
\renewcommand{\arccos}[1]{\oldarccos\!\left(#1\right)}
\let\oldarctan\arctan{}
\renewcommand{\arctan}[1]{\oldarctan\!\left(#1\right)}

% For log, ln, etc
\let\oldlog\log{}
\renewcommand{\log}[1]{\oldlog\!\left(#1\right)}
\let\oldln\ln{}
\renewcommand{\ln}[1]{\oldln\!\left(#1\right)}

% For Re, Im, etc
\let\oldRe\Re{}
\renewcommand{\Re}[1]{\oldRe\!\left\{#1\right\}}
\let\oldIm\Im{}
\renewcommand{\Im}[1]{\oldIm\!\left\{#1\right\}}

% For Laplace, ILaplace, etc
\newcommand{\laplace}[1]{\,\mathscr{L}\!\left\{#1\right\}}
\newcommand{\ilaplace}[1]{\,\mathscr{L}^{-1}\!\left\{#1\right\}}

% For integrals
\let\oldint\int{}
\NewDocumentCommand{\intidef}{m o}{\oldint{} #1 \IfValueT{#2}{\,\mathrm{d}#2}}
\newcommand{\intdef}[4]{\oldint_{#1}^{#2} #3 \,\mathrm{d}#4}

% Defining Units that are not in the SI base
\DeclareSIUnit\bar{bar}
\DeclareSIUnit\foot{ft}
\DeclareSIUnit\inch{in}
\DeclareSIUnit\year{yr}
\DeclareSIUnit\hour{hr}
\DeclareSIUnit\failure{failure}
\DeclareSIUnit\cycle{cycle}
\DeclareSIUnit\DU{DU}
\DeclareSIUnit\AU{AU}
\DeclareSIUnit\TU{TU}
\DeclareSIUnit\dollar{\$}
\DeclareSIUnit\cent{\text{\textcent}}
\DeclareSIUnit\c{\degreeCelsius}
\DeclareSIUnit\deg{\degree}
\DeclareSIUnit\rad{\radian}
\DeclareSIUnit\rev{rev}

% Griffiths script char LOL
\DeclareMathAlphabet{\mathcalligra}{T1}{calligra}{m}{n}
\DeclareFontShape{T1}{calligra}{m}{n}{<->s*[2.2]callig15}{}
\newcommand{\rcurs}{\mathcalligra{r}\,}
\newcommand{\brcurs}{\pmb{\mathcalligra{r}}\,}

% Hyperlink config
\hypersetup{
    colorlinks=true,
    linkcolor=blue,
    filecolor=magenta,
    urlcolor=blue,
    }
\urlstyle{same}

% Code Listing style
\definecolor{mintedbackground}{rgb}{0.95,0.95,0.95}
\definecolor{mintedframe}{rgb}{0.70,0.85,0.95}
\setminted{
	bgcolor=mintedbackground,
	linenos=true,
	numberblanklines=true,
	numbersep=12pt,
	numbersep=5pt,
	autogobble=true,
	frame=leftline,
	framesep=2mm,
	tabsize=4,
	breaklines=true,
	style=sas,
}

